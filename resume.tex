% template by zmanji/zameermanji.com
\documentclass[10.5pt,letterpaper]{article}
\usepackage[margin=0.75in]{geometry}
\usepackage[utf8]{inputenc}
\usepackage[T1]{fontenc}
\usepackage[stretch=10]{microtype}
\usepackage{hyperref}
\usepackage{tgpagella}
\usepackage{enumitem}
\usepackage{indentfirst}
\pagestyle{empty}

\linespread{1.10}
\begin{document}

\begin{center}
{\Large \textbf{Santi Santichaivekin}}

\ \ \href{mailto:jsantichaivekin@hmc.edu}{jsantichaivekin@hmc.edu}\ \ 
\ \ \textbullet
\ \ \href{https://github.com/ssantichaivekin}{github.com/ssantichaivekin}
\ \ \textbullet
\ \ Mobile : 347-401-3715

\end{center}


\hrule
\vspace{-0.95em}
\subsection*{Education}
  \begin{itemize}
    \parskip=-0.5em

    % Harvey Mudd College
    \item[]
    \textbf{Harvey Mudd College} \hfill
      Claremont, CA\\
    {B.S. in Computer Science, Major GPA 4.00 (Cumulative GPA 3.73) \hfill \emph{Class of 2021}}
  \end{itemize}
  \vspace{-2.07em}
\subsection*{Coursework}
\begin{itemize}
\item[]
    \parskip=-0.3em
  \textbf{In progress:}  Computer Systems, Science of Debugging, Phylogenic Tree Reconstruction (Independent Study)\\
  \textbf{Completed:} Data Structures and Program Development; 
  Multivariable Calculus;
  Intro to CS (Advanced); \\
  Computability and Logic; 
  Probability and Statistics;
  Linear Algebra;
  Differential Equations

\end{itemize}

\hrule
\vspace{-0.95em}
\subsection*{Skills}
\begin{itemize}
\item[]
    \parskip=-0.2em
  Proficient : C++, Python | Knowledgeable: C, C\#, Ruby, Java, JavaScript, Bash, Prolog, R, web development
    
\end{itemize}

\hrule
\vspace{-0.95em}
\subsection*{Experience}
  \begin{itemize}
    \parskip=-0.45em

    \item[]
    {\textbf{CS Tutor/Grader}, \textit{Harvey Mudd College}, Claremont, CA \hfill {Spring 2018 - Present}}
    \begin{itemize}[label=\textbullet]
      \itemsep0.1em
      \item Tutor and grade Computability and Logic class which teaches proof methods, automata, prolog,
      and computability theory. 
      \item Tutor and grade CS For Insight class which focused 
      on scripting and using python libraries for everyday tasks such as file management, web-scraping, 
      machine learning, and HTML generation. Used Python.
    \end{itemize}
    \vspace{0.1em}
    \item[]
    {\textbf{Software Engineer Intern}, \textit{Microsoft}, Bellevue, WA \hfill {Summer 2018}}
    \begin{itemize}[label=\textbullet]
      \itemsep0.1em
      \item Implemented an event queue to perform layout calculations in the background 
      across multiple frames. This makes complex visual transitions in Microsoft Whiteboard application
      more responsive. Used C\#.
    \end{itemize}
    \vspace{0.1em}
    \item[]
    {\textbf{Team Leader}, \textit{Thailand National Software Contest 2016}, Bangkok, Thailand \hfill {Spring 2016}}
    \begin{itemize}[label=\textbullet]
      \itemsep0.1em
      \item Arranged monthly meetings for 3-person team and headed application development.
      \item Designed a system for Thai word segmentation using maximum dictionary matching and exhaustive 
      search in JavaScript.
      \item Deployed a web application to analyze the analyze rythms, sounds, rhymes, and alliterations of Thai 
      poems using the International Phonetic Alphabet (IPA).
      
    \end{itemize}


  \end{itemize}
  
\hrule
\vspace{-0.95em}
\subsection*{Personal Projects}
  \begin{itemize}
    \parskip=-0.45em
    \item[]
    {\textbf{Halite3 AI Competition Bot} (\href{https://github.com/ssantichaivekin/halite3}
    {github.com/ssantichaivekin/halite3})\hfill {Fall 2018}}
    \begin{itemize}[label=\textbullet]
        \item  Written evaluation functions to plan ship movements and navigate them around the game map without colliding.
        Used Python and then switched to C++.
        \item Used Evolutionary Algorithm (Python) to fine-tune hyperparameters on DigitalOcean server.
        \item Finished with rank 201 out of 4014 total participants.
    \end{itemize}
    \vspace{0.1em}
    \item[]
    {\textbf{Text to Color Tone} (\href{https://github.com/ssantichaivekin/text-to-color-tone}
    {github.com/ssantichaivekin/text-to-color-tone})\hfill {Summer 2018}}
    \begin{itemize}[label=\textbullet]
        \item  Used Google Custom Search API and k-nearest neighbor algorithm to find a color tone 
        of any text and display it on screen using matplotlib python library.
        \item  Computed the color tone of different word category such as adjectives, nouns, verbs, 
        words that start with "a", and words with first vowel sound "ow", using NLTK dictionary to 
        group words and Digital Ocean cloud server to download images and perform calculations.
    \end{itemize}
    \vspace{0.1em}
    \item[]
    {\textbf{IPython Chemistry Calculator} (\href{https://github.com/ssantichaivekin/chem-calculator}
    {github.com/ssantichaivekin/chem-calculator})\hfill {Spring 2018}}
    \begin{itemize}[label=\textbullet]
        \item  Developed a small python library to help with chemistry calculations. 
        The project uses regular expression to read chemical formulas and BeautifulSoup to scrape 
        Wikipedia for mass and thermodynamics data.
    \end{itemize}
  \end{itemize}

\hrule
\vspace{-0.95em}
\subsection*{Honors and Awards}
  \begin{itemize}
    \parskip=-0.5em
    % Twitter 2.0
    \item[]
    {5\textsuperscript{th} place in ACM-ICPC Contest SoCal Region, 2018}\\
    {5\textsuperscript{th} place in ACM-ICPC Contest SoCal Region, 2017}\\
    {1\textsuperscript{st} place in Harvey Mudd College Microsoft Coding Competition, 2017}
  \end{itemize}

% \hrule
% \vspace{-1.0em}
% \subsection*{Hobbies}
% \begin{description}[labelindent=\parindent]
%   \parskip=-0.5em
% \item[] {Unicycle, Competitive Programming, Drawing, Painting, Subtitling Educational Videos.}
% \end{description}

\end{document}
