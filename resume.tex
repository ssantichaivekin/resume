% template by zmanji/zameermanji.com
\documentclass[10.5pt,letterpaper]{article}
\usepackage[margin=0.75in]{geometry}
\usepackage[utf8]{inputenc}
\usepackage[T1]{fontenc}
\usepackage[stretch=10]{microtype}
\usepackage{hyperref}
\usepackage{tgpagella}
\usepackage{enumitem}
\usepackage{indentfirst}
\pagestyle{empty}

\linespread{1.10}
\begin{document}

\begin{center}
{\Large \textbf{Santi Santichaivekin}}

\ \ \href{mailto:jsantichaivekin@hmc.edu}{jsantichaivekin@hmc.edu}\ \ 
\ \ \textbullet
\ \ \href{https://github.com/ssantichaivekin}{github.com/ssantichaivekin}
\ \ \textbullet
\ \ Mobile : 347-401-3715

\end{center}


\hrule
\vspace{-1.0em}
\subsection*{Education}
  \begin{itemize}
    \parskip=-0.5em

    % Harvey Mudd College
    \item[]
    \textbf{Harvey Mudd College} \hfill
      Claremont, CA\\
    {B.S. in Computer Science, Major GPA 4.00 (Cumulative GPA 3.60) \hfill \emph{Class of 2021}}
  \end{itemize}
  \vspace{-2.2em}
\subsection*{Coursework}
\begin{itemize}
\item[]
    \parskip=-0.2em
  Data Structures and Program Development \hfill Fall 2018\\
  CS-Principles and Practice, Computability and Logic \hfill 2017 - 2018

\end{itemize}

\hrule
\vspace{-1.0em}
\subsection*{Skills}
\begin{itemize}
\item[]
    \parskip=-0.2em
  \textbf{Programming} : Proficient : C++, Python | Knowledgeable: C\#, C, Ruby, Java, JavaScript, Bash\\
  \textbf{Language} : Thai (fluent)
    
\end{itemize}

\hrule
\vspace{-1.0em}
\subsection*{Experience}
  \begin{itemize}
    \parskip=-0.2em
    
    \item[]
    {\textbf{Software Engineer Intern}, \textit{Microsoft}, Bellevue, WA \hfill {May 18 - Jul 18}}
    \begin{itemize}[label=\textbullet]
      \itemsep0em
      \item Implemented a layout engine feature that perform layout calculations in the background across multiple frames. This feature allows for smoother and more responsive transition for complex visual elements in the Microsoft Whiteboard application.
      
    \end{itemize}
    {\textbf{CS Tutor/Grader}, \textit{Harvey Mudd College}, Claremont, CA \hfill {Jan 18 - Present}}
    \begin{itemize}[label=\textbullet]
      \itemsep0em
      \item Tutored and graded students' works for \textit{CS35: CS For Insight} class. The class focused on scripting and using python libraries for everyday tasks such as file management, web-scraping, basic machine learning, and HTML generation.
      
    \end{itemize}
    
    \item[]
    {\textbf{Team Leader}, \textit{National Software Contest 2016}, Bangkok, Thailand \hfill {Jan 16 - Jul 16}}
    \begin{itemize}[label=\textbullet]
      \itemsep0em
      \item Arranged monthly meetings for 3-person team and headed application development.
      \item Designed a system for Thai word segmentation using maximum dictionary Matching and exhaustive search in JavaScript.
      \item Deployed a JavaScript web application to analyze the rythm and sound of Thai poems using the international phonetic alphabet.
      
    \end{itemize}
    
    {\textbf{Member}, \textit{Bodindecha School Robotics Team}, Bangkok, Thailand \hfill {Jan 14 - Oct 16}}
     
    \begin{itemize}[label=\textbullet]
      \itemsep0em
      \item Implemented algorithms for robot movement in C and C++.
      \item Designed communication protocol for robot interactions via Bluetooth.
      \item Practiced weekly and competed in local competitions.
      
    \end{itemize}


  \end{itemize}
  
\hrule
\vspace{-1.0em}
\subsection*{Personal Projects}
  \begin{itemize}
    \parskip=-0.2em
    % Twitter 2.0
    \item[]
    {\textbf{Text to Color Tone} (\href{https://github.com/ssantichaivekin/text-to-color-tone}{github.com/ssantichaivekin/text-to-color-tone})\hfill {May 18 - Jul 18}}
    \begin{itemize}[label=\textbullet]
        \item  Used Google Custom Search API and k-nearest neighbor algorithm to find a color tone of any text and display it on screen using matplotlib python library.
        \item  Computed the color tone of different word category such as adjectives, nouns, verbs, words that start with "a", and words with first vowel sound "ow", using NLTK dictionary to group words and Digital Ocean cloud server to do download image information and perform calculations.
    \end{itemize}
    {\textbf{IPython Chemistry Calculator} (\href{https://github.com/ssantichaivekin/chem-calculator}{github.com/ssantichaivekin/chem-calculator})\hfill {Feb 18 - May 18}}
    \begin{itemize}[label=\textbullet]
        \item  Developed a small python library to help with chemistry calculations. The project uses regular expression to read chemical formulas and BeautifulSoup to scrape Wikipedia for mass and thermodynamics data.
    \end{itemize}
  \end{itemize}

\hrule
\vspace{-1.0em}
\subsection*{Honors and Awards}
  \begin{itemize}
    \parskip=-0.5em
    % Twitter 2.0
    \item[]
    {5\textsuperscript{th} place in ACM-ICPC Contest SoCal Region, 2017}\\
    {1\textsuperscript{st} place in Harvey Mudd College Microsoft Coding Competition, 2017}
  \end{itemize}

\hrule
\vspace{-1.0em}
\subsection*{Hobbies}
\begin{description}[labelindent=\parindent]
  \parskip=-0.5em
\item[] {Competitive Programming, Unicycling, Drawing, Painting, Subtitling Educational Videos.}
\end{description}

\end{document}
